% --- Template for thesis / report with tktltiki2 class ---
% 
% last updated 2013/02/15 for tkltiki2 v1.02

\documentclass[finnish]{tktltiki2}

% tktltiki2 automatically loads babel, so you can simply
% give the language parameter (e.g. finnish, swedish, english, british) as
% a parameter for the class: \documentclass[finnish]{tktltiki2}.
% The information on title and abstract is generated automatically depending on
% the language, see below if you need to change any of these manually.
% 
% Class options:
% - grading                 -- Print labels for grading information on the front page.
% - disablelastpagecounter  -- Disables the automatic generation of page number information
%                              in the abstract. See also \numberofpagesinformation{} command below.
%
% The class also respects the following options of article class:
%   10pt, 11pt, 12pt, final, draft, oneside, twoside,
%   openright, openany, onecolumn, twocolumn, leqno, fleqn
%
% The default font size is 11pt. The paper size used is A4, other sizes are not supported.
%
% rubber: module pdftex

% --- General packages ---

\usepackage[utf8]{inputenc}
\usepackage[T1]{fontenc}
\usepackage{lmodern}
\usepackage{microtype}
\usepackage{amsfonts,amsmath,amssymb,amsthm,booktabs,color,enumitem,graphicx}
\usepackage[pdftex,hidelinks]{hyperref}
\usepackage{url}
\usepackage{listings}

% Automatically set the PDF metadata fields
\makeatletter
\AtBeginDocument{\hypersetup{pdftitle = {\@title}, pdfauthor = {\@author}}}
\makeatother

% --- Language-related settings ---
%
% these should be modified according to your language

% babelbib for non-english bibliography using bibtex
\usepackage[fixlanguage]{babelbib}
\selectbiblanguage{finnish}

% add bibliography to the table of contents
\usepackage[nottoc]{tocbibind}
% tocbibind renames the bibliography, use the following to change it back
\settocbibname{Lähteet}

% --- Theorem environment definitions ---

\newtheorem{lau}{Lause}
\newtheorem{lem}[lau]{Lemma}
\newtheorem{kor}[lau]{Korollaari}

\theoremstyle{definition}
\newtheorem{maar}[lau]{Määritelmä}
\newtheorem{ong}{Ongelma}
\newtheorem{alg}[lau]{Algoritmi}
\newtheorem{esim}[lau]{Esimerkki}

\theoremstyle{remark}
\newtheorem*{huom}{Huomautus}


% --- tktltiki2 options ---
%
% The following commands define the information used to generate title and
% abstract pages. The following entries should be always specified:

\title{Ohjelmoitavat sävyttimet}
\author{Janne Timonen}
\date{\today}
\level{Seminaaritutkielma}
\abstract{Tiivistelmä.}

% The following can be used to specify keywords and classification of the paper:

\keywords{avainsana 1, avainsana 2, avainsana 3}

% classification according to ACM Computing Classification System (http://www.acm.org/about/class/)
% This is probably mostly relevant for computer scientists
% uncomment the following; contents of \classification will be printed under the abstract with a title
% "ACM Computing Classification System (CCS):"
% \classification{}

% If the automatic page number counting is not working as desired in your case,
% uncomment the following to manually set the number of pages displayed in the abstract page:
%
% \numberofpagesinformation{16 sivua + 10 sivua liitteissä}
%
% If you are not a computer scientist, you will want to uncomment the following by hand and specify
% your department, faculty and subject by hand:
%
% \faculty{Matemaattis-luonnontieteellinen}
% \department{Tietojenkäsittelytieteen laitos}
% \subject{Tietojenkäsittelytiede}
%
% If you are not from the University of Helsinki, then you will most likely want to set these also:
%
% \university{Helsingin Yliopisto}
% \universitylong{HELSINGIN YLIOPISTO --- HELSINGFORS UNIVERSITET --- UNIVERSITY OF HELSINKI} % displayed on the top of the abstract page
% \city{Helsinki}
%


\begin{document}

% --- Front matter ---

\frontmatter      % roman page numbering for front matter

\maketitle        % title page
\makeabstract     % abstract page

\tableofcontents  % table of contents

% --- Main matter ---

\mainmatter       % clear page, start arabic page numbering

\section{Johdanto}

% Write some science here.

\emph{Sävyttimet} ovat ohjelmia, joiden tehtävänä \emph{grafiikkaliukuhihnalla} (asteittain etenevä prosessi tuottaa kuvia näytettäväksi) on \emph{sävyttää}, eli tuottaa tietyillä tavoilla dataa kuvaksi. Tämä voi tarkoittaa esimerkiksi jonkin objektin piirtämistä sijainnin mukaan, per-pikseli -värinmääritystä, pinnanmuotojen simulointia tai muita erikoistehostemaisiakin keinoja. Sävytin ottaa syötteenään elementin, esimerkiksi monikulmion \emph{kärkipisteen}, \emph{primitiivin}, eli monikulmion kuten kolmion, tai \emph{fragmentin}, kuten pikselin, ja tuottaa syötteestä tuloksena muunnettuja elementtejä, joiden määrä voi vaihdella nollasta useaan, riippuen sävyttimestä ja sen suorittamasta tehtävästä \cite{Gre14}.

Tutkielmassa tarkastellaan aluksi hieman yleisesti 3D-grafiikkaa, jotta sävyttimien roolia grafiikan tuottamisessa voi ymmärtää paremmin ja hahmottaa niiden tehtävää, minkä jälkeen käydään läpi sävyttimien historian päävaiheet, ja kuinka nykyisiin ohjelmoitaviin sävyttimiin on päädytty. Tämän jälkeen siirrytään asian varsinaiseen ytimeen, eli ohjelmoitaviin sävyttimiin, ja käydään vaiheittain läpi tällä hetkellä käytettävien sävyttimien toimintaa, mitä niillä on esimerkiksi mahdollista tehdä ja kuinka se tapahtuu.

\section{3D-grafiikka}

3D-grafiikkassa. ja erityisesti tämän tutkielman tapauksessa \emph{rasterointia}, eli kuvan muuttamista pikseleillä esitettävään muotoon (rasterikuvaksi), hyödyntävässä 3D-grafiikan reaaliaikaisessa renderoinnissa tuotetaan kolmiulotteisista malleista ja asioista kaksiulotteinen representaatio, eli esimerkiksi katsojan näkemä kuva tietokoneen ruudulla. Tarkkaa reaaliaikaisuutta vaativien grafiikkasovellusten, kuten pelien, tapauksessa nopea kuvan piirtäminen nousee tärkeäksi vaatimukseksi, jolloin monikulmioista, eli \emph{polygoneista}, tuotetaan rasteroimalla kaksiulotteista kuvaa tavoitteena ruudunpäivitysnopeus, joka vaikuttaa ihmisen silmään sulavalta (noin 30 ruutua sekunnissa (30 \emph{fps}) \cite{Gre14}. 

Ei-reaaliaikaisiin 3D-grafiikan renderointitapoihin lukeutuu esimerkiksi \emph{säteenjäljitys} (ray tracing), jossa esimerkiksi valaistus on globaalia, eli sisäisesti jo olemassa 3D-mallissa, ennen mahdollista projektiota kaksiulotteiseksi kuvaksi \cite{Puh08}. Tällainen renderointi on hidasta, eikä vielä tällä hetkellä ole järkevästi hyödynnettävissä reaaliaikaisessa käytössä, kuten peleissä, mutta ennakkoon renderoituna tuottaa lähes fotorealistista kuvaa. Säteenjäljitykseen, tai vastaaviin tekniikoihin, ei tässä tutkielmassa enää palata.

\subsection{Grafiikkaliukuhihna}
Rasterointiin tähtäävässä 3D-grafiikassa hyödynnetään niin kutsuttua liukuhihnaa (pipeline), joka on pääasiassa sarja tietyssä järjestyksessä tehtäviä askeleita, tai työvaiheita, joilla 3D-malleista luodaan 2D-rasterikuva. Liukuhihna, ja erityisesti \emph{grafiikkaliukuhihna} on siis prosessi, jolla lopullinen kuva tuotetaan. Tähän tarkoitukseen suosituimmat grafiikkaohjelmointirajapinnat (API) ovat OpenGL ja Direct3D.

Sävyttimien ymmärtämisen kannalta on hyödyllistä ymmärtää myös erilaiset 3D-grafiikkaan liittyv't \emph{koordinaatistot}, joissa 3D-malleja käsitellään. Näihin lukeutuvat muun muassa \emph{mallikoordinaatisto} (voidaan puhua myös koordinaatiston sijaan avaruudesta, esim \emph{model space}, jossa ei kyseisen 3D-mallin lisäksi ole mitään muuta, ja \emph{maailmakoordinaatisto}, jossa malli taas esiintyy suhteessa muihin malleihin, jotka maailmakoordinaatistoon on kytketty.

\section{Sävyttimien historiaa}

Ennen ensimmäisiä \emph{grafiikkakiihdyttimiä}, joihin kuului muun muassa 3Dfx:n Voodoo, grafiikan renderöinti tapahtui prosessorilla (CPU), jonka työtaakkaa erilliset grafiikkapiirit kehitettiin vähentämään. Ennen ohjelmoitavia sävyttimiä grafiikkaa tuotettiin käyttämällä hyväksi näytönohjaimien (GPU) \emph{kiinteää liukuhihnaa} (Fixed-Function Pipeline). Kehittäjä saattoi siis antaa raskaat laskutyöt näytönohjaimelle, tai erilliselle kiihdytinkortille, hoidettavaksi kiinteällä liukuhihnalla, mutta itse liukuhihnan suorittamiin funktioihin ei voinut puuttua muuten kuin parametrien avulla. Kiinteä liukuhihna näytönohjaimessa nopeutti laskentaa, ja toi mukanaan mahdollisuuksia luoda standardioperaatioiden rajoissa graafisia tehokeinoja ja efektejä, esimerkiksi Goraud-sävytyksen, josta esimerkki kuvassa \ref{FGP}.

\begin{figure}[!htbp]
\includegraphics[width=\textwidth]{/Users/jannetim/Documents/SeminaariF2015materials/3-shadings-flat-gouraud-phong.jpg}
\caption{Tasainen, Gouraud- ja Phong-sävytys}
\label{FGP}
\end{figure}

Myöhemmin tulivat ensimmäiset \emph{ohjelmoitavaa renderointiliukuhihnaa} tukevat näytönohjainpiirit, joissa kiinteän liukuhihnan pystyi korvaamaan omilla vapaasti ohjelmoitavilla sävyttimillä. Korvaamalla kiinteän liukuhihnan laskenta nykyajan grafiikkapiirien tukemilla ohjelmoitavilla sävyttimillä saavutetaan vapaus muokata vapaasti laskenta- ja muokkausoperaatioita, mikä antaa mahdollisuuden piirtää kuvaa enemminkin luovuuden rajoissa, kuin ennaltamääriteltyjen ehtojen. Ensimmäiset sävytinmallit tukivat ainoastaan alemman tason konekielillä ohjelmointia. Sen lisäksi, että kehittäjien täytyi luoda sävyttimen konekielellä, täytyi sävytin luoda lisäksi usein erikseen sekä OpenGL- että Direct3D-rajapinnoille johtuen näiden kahden suosituimman rajapinnan konekielien poikkeavuuksista. Myöhemmin verteksi-, eli kärkipiste-, ja pikselisävyttimet alkoivat yleistyä. Sävyttimien käyttö grafiikkaliukuhihnalla mahdollistaa rinnakkaistamisen erittäin hyvin. 
\cite{Ake02}

Eräs edelläkävijöitä sävyttimien saralla oli tietokoneanimaatioelokuvistaan tunnettu Pixar-yhtiö kehittämällään \emph{RenderMan}-kielellä, jota käytettiin muun muassa Toy Story -elokuvan tuottamiseen.

\section{Korkean tason sävytinkielet}

Ohjelmoitavien sävyttimien alkuaikoina oli sävyttimien luomiseen siis mahdollista käyttää vain alemman tason konekieliin pohjautuvia sävytinkieliä. Korkean tason kielillä on useita hyötyjä konekieliin nähden, ja ne pätevät myös korkean tason sävytinohjelmoinnissa: helpompi luettavuus, kirjoitettavuus, muokattavuus, virheiden etsintä ja löytäminen sekä yleisesti kehitysvauhdin nopeus \cite{She08}. Ohjelmoitavien sävyttimien tultua kasvoi myös tarve korkean tason sävytinkielille, joista mainittavimpina muodostuivat C-pohjaiset Nvidian Cg (C for Graphics) \cite{Nvi03} ja Microsoftin HLSL (High Level Shading Language) -kielet, jotka syntyivät samasta yhteistyöprojektista, ja ovatkin hyvin samankaltaiset, sekä OpenGL:n GLSL -kieli \cite{Ope15}. Näistä kolmesta Cg on jo deprekoitunut. Tässä tutkielmassa esimerkkikielenä käytetään pääasiassa Direct3D:n HLSL:ää.

\section{Ohjelmoitavat sävyttimet}

Ohjelmoitavat sävyttimet antavat algoritmeillaan mahdollisuuden muokata vapaasti ja reaaliaikaisesti kuvaa muodostaviin elementteihin liittyviä attribuutteja. Yksi sävytinvaihe ottaa syötteenään vastaan edellisen tulosteen, joten jokainen vaihe voi jatkaa seuraavan datan työstämistä, kun on saanut edellisen työn valmiiksi. Ohjelmoitavien sävyttimien osalta liukuhihna rakentuu tällä hetkellä pääpiirteissään \emph{kärkipiste}-, \emph{tesselaatio}, \emph{geometria}- ja \emph{pikseli}sävyttimestä. Kärkipistesävytin antaa laskutuloksensa (vaihtoehtoiselle) tesselaatiosävyttimelle, joka antaa tuloksensa (vaihtoehtoiselle) geometriasävyttimelle, joka antaa puolestaan tuloksensa pikselisävyttimelle. Tesselaatiosävytin ja geometriasävytin ovat uudempia tulokkaita, ja niiden käyttö ei ole pakollista, jolloin nämä efektit voidaan jättää pois tai korvata muulla tavoin, esimerkiksi ennen tesselaatiolle omistettua omaa sävytintä tesselaatiosävytys toteutettiin oman kiinteän vaiheen ja geometriasävyttimen avuin.

Sävyttimien käyttö mahdollistaa myös hyvin rinnakkaisuuden käytön, kun muunnoksia tehdään suurille datamäärille kerrallaan, esimerkiksi kaikille ruudun pikseleille. Moderneille grafiikkapiireillä onkin useita sävytinliukuhihnoja rinnakkaisuusmahdollisuuksien hyödyntämiseksi.

Tässä luvussa tarkastellaan erilaisia ohjelmoitavia sävyttimiä siinä järjestyksessä, jossa ne toimivat grafiikkaliukuhihnalla.

\subsection{Kärkipistesävyttimet}

\emph{Kärkipistesävytin}, tai \emph{verteksisävytin}, ajetaan kerran jokaista monikulmion, tarkemmin kolmion, kärkipistettä kohden. Kärkipistesävytin ottaa syötteenään kärkipisteen \emph{attribuuttitiedon}, joka sisältää muun muassa kyseisen kärkipisteen sijainnin malli- tai maailmakoordinaatistossa, sekä pinnan normaalivektorin. Tulosteena kärkipistesävytin antaa yhden kärkipisteen, joka on käynyt läpi valaistuksen ja koordinaatiston muunnosvaiheet, ja joka ilmaistaan \emph{normalisoidussa kuvausavaruudessa}. Yleisesti siis kärkipistesävytin voi muokata monikulmion kärkipisteen, normaalin, tekstuurikoordinaattien ja paikan arvoja. Sävyttimellä voi luoda esimerkiksi tuulessa heiluvat puiden oksat tai vedenpinnan väreily.

Kärkipistesävyttimen tärkeimpiin tehtäviin kuuluu siis tehdä tarvittavat koordinaatistomuunnokset syötteinä saaduille kärkipisteille. Ensimmäisenä tehdään \emph{mallimuunnosa}, eli muunnos mallikoordinaatistosta maailmakoordinaatistoon, jolloin kärkipiste sidotaan omasta paikallisesta koordinaatistostaan absoluuttisesti maailmanäkymään. Tämän jälkeen tehdään \emph{katsojamuunnos} maailmakoordinaatistosta katsojan koordinaatistoon, jolloin origo on katsotaan katsojan ``kameran'' näkökulmasta. Viimeiseksi suoritetaan \emph{projektio-} tai \emph{perspektiivimuunnos}, jolloin tuotetaan renessanssiajan kuvataiteesta tuttu luonnollisen elämän persepektiiviä jäljittelevä kuva, jossa kauempana olevat kohteet näkyvät pienempinä katsojan \emph{näköfrustumissa}. Tätä tilaa kutsutaan normalisoiduksi kuva-avaruudeksi \cite{Puh08}.Vähintään kärkipistesävyttimen tulee siis antaa tuloksena kärkipiste muunnettuna normalisoituun kuva-avaruuteen \emph{uniformina}, eli vakiomuotoisena, tietona. \cite{Puh08}

Kärkipistesävytin on pakollinen vaihe grafiikkaliukuhihnalla, ja sen täytyy vähintään kuljettaa lävitseen syötteenä saatu kärkipiste, vaikkei 

\begin{lstlisting}
 struct VSInput
 {
	 float4 Pos : POSITION;
	 float3 Normal : NORMAL;
	 float2 Texcoord : TEXCOORD0;
 };
 
 PSInput VertexShader(VSInput In)
 {
	 PSInput Out;	
	
	 Out.Normal = mul(In.Normal, (float3x3)g_mWorld);
	 Out.WorldPos = mul(In.Pos, g_mWorld);	
	 Out.Pos = mul(Out.WorldPos, g_mViewProj);	
	 Out.Texcoord = In.Texcoord;		
		
	 return Out;
 }
\end{lstlisting}

\subsection{Tesselaatiosävytin}

Tutkielman sävyttimistä uusin on Shader Model 5.0:n, rajapintojen kohdalla DirectX 11 ja OpenGL 4.0, myötä tullut tesselaatiosävytin. \emph{Tessellaatiossa} primitiivi, \emph{monikulmioverkko} (polygon mesh), eli kärkipisteistä ja reunaviivoista kohtaava kolmioiden joukko\cite{Puh08}, jaetaan pienempiin osasiin, kuten vaikkapa kolmio kahteen pienempään kolmioon \cite{Nvi10}. Yksinkertaisesti siis tesselaatio on monikulmioiden rikkomista ja jakamista pienempiin ja hienompiin osasiin. 

Tessellaatiosävytin on jaettu kolmeen vaiheeseen, joista ensimmäinen ja kolmas ovat ohjelmoitavia sävytinvaiheita: \emph{Hull Shader}, traditionaalinen tesselaatiovaihe\emph{Tessellation Stage} ja \emph{Domain Shader}   \cite{Mic11}

Pelkkänä menetelmänä tessellointi ei välttämättä tunnu tuovan mitään mullistavaa esimerkiksi pelien ulkonäköön, sillä ulkoasun kannalta ei ole merkitystä onko esimerkiksi neliö renderoitu kahden vai satojen kolmioiden avulla. Sen sijaan yhdistämällä tesselaatioon muita tekniikoita, ja laittamalla monikulmioista pilkotut palaset esittämään uutta informaatiota, saadaan graafista esitystä realistisemmaksi \cite{Nvi10}. 

Eräs tekniikka on \emph{nyrjäyttäminen} (Displacement mapping), jossa tesseloidun pinnan kärkipisteitä nostetaan tai lasketaan korkeusattribuutin perusteella, jolloin saadaan luotua epätasaisia pintoja \cite{Nvi10}. Tessellointi säästää myös muistia ja kaistanleveyttä mahdollistamalla yksityiskohtaiset pinnat pieniresoluutioisilla tekstuureilla \cite{Mic11} \cite{Nvi10}.

Tesselaation avulla saavutettava keino on myös 3D-mallien silottaminen PN(point normal)-kolmioiden avulla \cite{Vla01}. Esimerkki kuvassa \ref{SCoP} näkyvän Stalker: Call of Pripyat -pelin hahmon kaasunaamarin suodattimen reunoja on saatu tesselaatiolla luonnollisemmalla tavalla kaartuviksi verrattuna yksinkertaisempaan ja vähäisemmillä monikulmioilla piirrettyyn malliin. 

\begin{figure}[!htbp]
\includegraphics[width=\textwidth]{/Users/jannetim/Documents/SeminaariF2015materials/stalkercop-dx11.jpg}
\caption{Monikulmiomallin silottaminen tesselaation avulla pelissä Stalker: Call of Pripyat}
\label{SCoP}
\end{figure}

\emph{Dynaamisella tesselaatiolla} voidaan esimerkiksi skaalata mallien piirron tarkkuutta näkyvyyden suhteen muuttamalla yksityiskohtien määrää lennosta \cite{Nvi10}. Tällöin esimerkiksi avarassa ulkoilmapelinäkymässä kaukaa katsottuna jostain mallista piirretään malli muutamalla monikulmiolla, ja lähestyttäessä monikulmioiden määrää lisätään dynaamisesti, kunnes läheltä katsottuna malli voidaan piirtää tuhansista monikulmioista \cite{Gre14}.

\subsection{Geometriasävyttimet}

Geometriasävyttimet on kärkipiste- ja pikselisävyttimiin verrattuna uudempi sävytin sen tultua esitellyksi DirectX 10:n myötä. Geometriasävytin sijaitsee renderointiliukuhihnalla tesselaatiosävyttimen jälkeen, ja ennen pikselisävytintä. Sen käyttö ei ole pakollista, ja vielä esimerkiksi pelien sävytyksessä on otettava huomioon jonkinlainen varakeino mikäli geometriasävyttimien käyttö ei ole mahdollista. Ennen tesselaatiosävyttimen tuloa geometriaävyttimellä hoidettavia tyypillisiä käyttötapauksia oli esimerkiksi aiemmin esitelty dynaaminen tesselaatio.

Geometriasävytin ottaa syötteenään \emph{n}-kärkipisteestä muodostuvia primitiiveja, kuten pisteitä (n=1), suoria (n=2) tai kolmioita (n=3). Syötteistä geometriasävytin muokkaa, valikoi ja jopa luo uusia primitiivejä \cite{Gre14}. Tuloksena voi siis olla nollasta useampaan primitiiviä, jotka eivät välttämättä ole samaa tyyppiä kuin syötteenä saadut. Geometriasävytin voi esimerkiksi yhdistellä kolmioita yhteen, tai hylätä kokonaan. Toisin kuin kärkipistesävytin, joka kykenee käsittelemään vain yhden monikulmion kärkipisteen kerrallaan, pystyy geometriasävytin ``näkemään'' koko primitiivin kaikkine kärkipisteineen.

\subsection{Pikseli-/fragmenttisävyttimet}

Pikselisävytin, tai fragmenttisävytin (riippuen sävytinkielen terminologiasta; myöhemmin tekstissä puhutaan HLSL:n konvention mukaan pikselisävyttimestä), on graafinen funktio, joka laskee muunnoksia per-pikseli -periaatteella, eli muunnokset voidaan tehdä jokaiselle yksittäiselle pikselille, tai muulle fragmentille, erikseen \cite{}. Pikselisävytin ajetaan kerran per pikseli, ja useita kertoja jokaista syötteenä saatua monikulmiota kohden, sillä sävytin käsittelee jokaista monikulmion pikseliä erikseen. Pikselin väriarvo sekä Z-syvyys lasketaan syötteenä saatun vektorimuotoisen datan, kuten normaalivektorin, värin, tekstuurikoordinaattien, interpoloitujen valonlähteiden suuntien ja katsojan suunnan, perusteella. Erityisesti pikseliin kohdistuva valaistuksen laskenta voidaan johtaa edellisistä \cite{Puh08}.

Monet näyttävät 3d-peleissä käytettävät tehostekeinot, kuten pinnan kuhmutus tai \emph{Fresnel-heijastus}, luodaan juuri pikselisävyttimien tasolla. Myös kuvan \ref{FGP} Phong-sävytys luodaan pikselisävyttimellä. Pikselisävyttimen tärkeimpiin tehtäviin lukeutuvatkin teksturointi ja valaistuksen laskenta.




% --- References ---
%
% bibtex is used to generate the bibliography. The babplain style
% will generate numeric references (e.g. [1]) appropriate for theoretical
% computer science. If you need alphanumeric references (e.g [Tur90]), use
%
% \bibliographystyle{babalpha-lf}
%
% instead.

%\bibliographystyle{babalpha-lf}
\bibliographystyle{babalpha-lf}
\bibliography{references-fi}


% --- Appendices ---

% uncomment the following

% \newpage
% \appendix
% 
% \section{Esimerkkiliite}

\end{document}