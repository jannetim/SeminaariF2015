% --- Template for thesis / report with tktltiki2 class ---
% 
% last updated 2013/02/15 for tkltiki2 v1.02

\documentclass[finnish]{tktltiki2}

% tktltiki2 automatically loads babel, so you can simply
% give the language parameter (e.g. finnish, swedish, english, british) as
% a parameter for the class: \documentclass[finnish]{tktltiki2}.
% The information on title and abstract is generated automatically depending on
% the language, see below if you need to change any of these manually.
% 
% Class options:
% - grading                 -- Print labels for grading information on the front page.
% - disablelastpagecounter  -- Disables the automatic generation of page number information
%                              in the abstract. See also \numberofpagesinformation{} command below.
%
% The class also respects the following options of article class:
%   10pt, 11pt, 12pt, final, draft, oneside, twoside,
%   openright, openany, onecolumn, twocolumn, leqno, fleqn
%
% The default font size is 11pt. The paper size used is A4, other sizes are not supported.
%
% rubber: module pdftex

% --- General packages ---

\usepackage[utf8]{inputenc}
\usepackage[T1]{fontenc}
\usepackage{lmodern}
\usepackage{microtype}
\usepackage{amsfonts,amsmath,amssymb,amsthm,booktabs,color,enumitem,graphicx}
\usepackage[pdftex,hidelinks]{hyperref}
\usepackage{url}
\usepackage{listings}

% Automatically set the PDF metadata fields
\makeatletter
\AtBeginDocument{\hypersetup{pdftitle = {\@title}, pdfauthor = {\@author}}}
\makeatother

% --- Language-related settings ---
%
% these should be modified according to your language

% babelbib for non-english bibliography using bibtex
\usepackage[fixlanguage]{babelbib}
\selectbiblanguage{finnish}

% add bibliography to the table of contents
\usepackage[nottoc]{tocbibind}
% tocbibind renames the bibliography, use the following to change it back
\settocbibname{Lähteet}

% --- Theorem environment definitions ---

\newtheorem{lau}{Lause}
\newtheorem{lem}[lau]{Lemma}
\newtheorem{kor}[lau]{Korollaari}

\theoremstyle{definition}
\newtheorem{maar}[lau]{Määritelmä}
\newtheorem{ong}{Ongelma}
\newtheorem{alg}[lau]{Algoritmi}
\newtheorem{esim}[lau]{Esimerkki}

\theoremstyle{remark}
\newtheorem*{huom}{Huomautus}


% --- tktltiki2 options ---
%
% The following commands define the information used to generate title and
% abstract pages. The following entries should be always specified:

\title{Ohjelmoitavat sävyttimet}
\author{Janne Timonen}
\date{\today}
\level{Seminaaritutkielma}
\abstract{Seminaarityö käsittelee ohjelmoitavia sävyttimiä, jotka ovat erityisesti tietokonepelien 3D-grafiikan tuotossa tehokkaita välineitä määritellä grafiikan piirtoon liittyviä ominaisuuksia ja luoda erilaisia graafisia tehokeinoja. Nykymuotoonsa sävyttimet ovat kehittyneet ennalta määritettyjen standardioperaatioiden eli niin kutsutun kiinteän liukuhihnan kautta. Nykyään tietokonepeleihin sävyttimiä ohjelmoidaan pääasiassa kahden suosituimman grafiikkaohjelmointirajapinnan (OpenGL ja Direct3D) sävytinkielillä, jotka ovat GLSL ja HLSL. Tekstissä tarkastellaan Shader Model 5.0 mukaisen grafiikkaliukuhihnan ohjelmoitavia sävyttimiä sekä liukuhihnan ulkopuolista laskentasävytintä.}

% The following can be used to specify keywords and classification of the paper:

\keywords{3D-grafiikka, ohjelmoitavat sävyttimet, reaaliaikainen grafiikka}

% classification according to ACM Computing Classification System (http://www.acm.org/about/class/)
% This is probably mostly relevant for computer scientists
% uncomment the following; contents of \classification will be printed under the abstract with a title
% "ACM Computing Classification System (CCS):"
% \classification{}

% If the automatic page number counting is not working as desired in your case,
% uncomment the following to manually set the number of pages displayed in the abstract page:
%
% \numberofpagesinformation{16 sivua + 10 sivua liitteissä}
%
% If you are not a computer scientist, you will want to uncomment the following by hand and specify
% your department, faculty and subject by hand:
%
% \faculty{Matemaattis-luonnontieteellinen}
% \department{Tietojenkäsittelytieteen laitos}
% \subject{Tietojenkäsittelytiede}
%
% If you are not from the University of Helsinki, then you will most likely want to set these also:
%
% \university{Helsingin Yliopisto}
% \universitylong{HELSINGIN YLIOPISTO --- HELSINGFORS UNIVERSITET --- UNIVERSITY OF HELSINKI} % displayed on the top of the abstract page
% \city{Helsinki}
%


\begin{document}

% --- Front matter ---

\frontmatter      % roman page numbering for front matter

\maketitle        % title page
\makeabstract     % abstract page

\tableofcontents  % table of contents

% --- Main matter ---

\mainmatter       % clear page, start arabic page numbering

\section{Johdanto}

% Write some science here.

\emph{Sävyttimet} ovat ohjelmia, joiden tehtävänä \emph{grafiikkaliukuhihnalla} (asteittainen kuvantuottoprosessi) on \emph{sävyttää} eli tuottaa tietyillä tavoilla dataa kuvaksi. Tämä voi tarkoittaa esimerkiksi jonkin objektin piirtämistä sijainnin mukaan, pikselikohtaista värinmääritystä, pinnanmuotojen simulointia tai muita erikoistehostemaisiakin keinoja. 

Sävytin ottaa syötteenään elementin, kuten esimerkiksi monikulmion kärkipisteen, kokonaisen monikulmion tai yksittäisen pikselin. Saadusta syötteestä sävytin tuottaa tuloksena muunnettuja elementtejä, joiden määrä voi vaihdella nollasta useaan, riippuen sävyttimestä ja sen suorittamasta tehtävästä \cite[s. 500]{Gre14}. Sävyttimien käyttö mahdollistaa myös hyvin rinnakkaisuuden käytön, kun muunnoksia tehdään suurille datamäärille kerrallaan, esimerkiksi kaikille ruudun pikseleille. Moderneilla grafiikkapiireillä onkin useita sävytinliukuhihnoja rinnakkaisuusmahdollisuuksien hyödyntämiseksi.

Tutkielmassa tarkastellaan aluksi hieman yleisesti 3D-grafiikkaa, jotta sävyttimien roolia grafiikan tuottamisessa voi ymmärtää paremmin ja hahmottaa niiden tehtävää, minkä jälkeen käydään läpi sävyttimien historian päävaiheet, ja kuinka nykyisiin ohjelmoitaviin sävyttimiin on päädytty. Tämän jälkeen siirrytään asian varsinaiseen ytimeen eli ohjelmoitaviin sävyttimiin, ja käydään vaiheittain läpi tällä hetkellä käytettävien sävyttimien toimintaa, mitä niillä on mahdollista tehdä ja kuinka se tapahtuu. Lopuksi tarkastellaan hieman tarkemmin kuinka sävytin voidaan kääntää ja sitoa mukaan grafiikan tuottoon.

\section{3D-grafiikka}

3D-grafiikassa tuotetaan kolmiulotteisista malleista kaksiulotteinen representaatio eli esimerkiksi katsojan näkemä kuva tietokoneen ruudulla. Erityisesti tämän tutkielman tapauksessa 3D-grafiikan reaaliaikaisen piirron menetelmänä käsitellään \emph{rasterointia} eli kuvan muuttamista pikseleillä esitettävään muotoon niin kutsutuksi rasterikuvaksi eli kaksiulotteiseksi pikseliruudukoksi. Rasteroinnissa kolmiulotteiset mallit on projisoitu ja leikattu piirtoikkunaan halutulla tavalla, minkä jälkeen kuvaa piirretään kaksiulotteiseen niin kutsuttuun rasteriruudukkuun, joka käytännössä koostuu pikseleistä. Tarkkaa reaaliaikaisuutta vaativien grafiikkasovellusten, kuten pelien, tapauksessa nopea kuvan piirtäminen nousee tärkeäksi vaatimukseksi, jolloin monikulmioista eli \emph{polygoneista} tuotetaan rasteroimalla kaksiulotteista kuvaa. Tavoitteena on saavuttaa ruudunpäivitysnopeus, joka vaikuttaa ihmisen silmään sulavalta, eli ainakin noin 30 ruutua sekunnissa. \cite[s. 444-445]{Gre14}. 

Ei-reaaliaikaisiin 3D-grafiikan renderointitapoihin lukeutuu esimerkiksi \emph{säteenjäljitys} (Ray tracing), jossa valaistus ja valon heijastukset lasketaan simuloiden tarkemmin todellista maailmaa. Esimerkiksi voidaan laskea pinnalta heijastuneen valon heijastuminen uudestaan joltain toiselta pinnalta \cite[s.405-406]{Puh08}. Tällainen piirto on hidasta, eikä vielä tällä hetkellä ole järkevästi hyödynnettävissä reaaliaikaisessa käytössä, kuten peleissä. Ennakkoon prosessoituna piirto silti tuottaa lähes fotorealistista kuvaa. Säteenjäljitykseen tai vastaaviin tekniikoihin ei tässä tutkielmassa enää palata.

Sävyttimien ymmärtämisen kannalta on hyödyllistä tuntea myös erilaiset 3D-grafiikkaan liittyvät \emph{koordinaattijärjestelmät} (tai avaruudet), joissa 3D-malleja käsitellään. Näihin lukeutuvat muun muassa \emph{malli-}, \emph{maailma-}, \emph{näkymä-} ja \emph{projektiokoordinaatisto}. Mallikoordinaatisto pitää sisällään vain jonkin tietyn 3D-mallin, esimerkiksi yksittäisen puun tai hahmon. Maailmakoordinaatisto käsittää avaruuden, johon objektit siirretään mallikoordinaatistosta ja jossa ne sijaitsevat absoluuttisesti tietyllä paikalla. Näkymäkoordinaatisto on kameran näkökulmasta, ja projektiokoordinaatistossa näkymään on lisätty perspektiivi. Sävyttimet siirtyvät koordinaatistosta toiseen tekemällä \emph{muunnoksia} eli tekemällä matriisilaskutoimituksia kulloiseenkin sijaintikoordinaatistoon \cite[s. 167-171]{Puh08}. Muunnoksista koordinatistojen välillä käsitellään tarkemmin kärkipistesävyttimestä kertovassa luvussa.

\subsection{Grafiikkaliukuhihna}
Rasterointiin tähtäävässä 3D-grafiikassa hyödynnetään niin kutsuttua liukuhihnaa (pipeline), joka on pääasiassa sarja tietyssä järjestyksessä tehtäviä askeleita tai työvaiheita, joilla 3D-malleista luodaan 2D-rasterikuva. Liukuhihna ja erityisesti \emph{grafiikkaliukuhihna} on siis prosessi, jolla lopullinen kuva tuotetaan. Peligrafiikan tuottamiseen rasteroinnin keinoilla suosituimmat grafiikkaohjelmointirajapinnat ovat OpenGL ja Direct3D, jotka kummatkin toimivat saman grafiikkaliukuhihna-ajattelun mukaisesti \cite{Krh15} \cite{Mic11}.

Valmistajien grafiikkapiirien tukemia sävyttimiä eritellään \emph{Shader Model} -versioinnin avulla, joka kertoo millaista grafiikkaliukuhihnaa ja mitä sävyttimiä jokin laite tukee. Tutkielman tarkastelema grafiikkaliukuhihna on Shader Model 5.0:n mukainen, jolloin voidaan esitellä kaikki nykyään peleissä mahdollisesti käytössä olevat sävyttimet. Grafiikkaliukuhihna koostuu kärkipiste-, tesselaatio-, geometria- ja pikselisävyttimestä.

Grafiikkaliukuhihnaan kuuluu myös joitain kiinteitä vaiheita, jotka eivät varsinaisesti ole sävyttimiä tai ainakaan ohjelmoitavia sellaisia. Esimerkiksi ennen varsinaista grafiikkapiirissä suoritettavia vaiheita on prosessorilla suoritettava sovellusvaihe, jossa esimerkiksi 3D-mallit valmistellaan sellaiseen muotoon, että grafiikkapiirtojärjestelmä voi niitä käyttää \cite[s. 164-165]{Puh08}. Geometriasävyttimen jälkeinen vaihtoehtoinen tulostevirta (Stream output stage) taas voi ohjata primitiividatan takaisin liukuhihnan aiempiin vaiheisiin. Rasterointivaiheessa puolestaan hoidetaan rasterointi. Näihin kiinteisiin vaiheisiin ei myöhemmin enää juurikaan paneuduta lukuun ottamatta tilanteita, joissa ne ovat sävyttimen käytön kannalta oleellisia. Liukuhihnan vaiheet ja järjestys ovat esitetty kuvassa \ref{gpipe} \cite{Mic11}.

\begin{figure}[!htb]
\centering
\includegraphics[scale=0.37]{/Users/jannetim/Documents/SeminaariF2015materials/DirectX-11-Rendering-Pipeline.png}
\caption{Grafiikkaliukuhihna}
\label{gpipe}
\end{figure}

\section{Sävyttimien historiaa}

1990-luvulta alkaen grafiikanpiirto on käynyt läpi erilaisia vaiheita. Laitteistotasolla on edetty harppauksin, mutta myös keinot tuottaa kuvaa ovat muuttuneet. Prosessorilla tehtävästä laskentatyöstä on edetty erillisten 3D-grafiikkakiihdyttimien kautta nykyisenkaltaisiin grafiikkapiireihin, joiden laskentateho ja tuki erilaisille teknologioille on moninkertainen verrattuna varhaisemman sukupolven laitteisiin. Sävyttimien kohdalla ohjelmointitavat ovat kulkeneet muokkaamattoman kiinteän liukuhihnan ja vaivalloisen konekielityylisen ohjelmoinnin kautta nykyisiin korkean tason sävytinkielisiin ohjelmiin. Tässä luvussa tarkastellaan hieman sävyttimien historian pääkohtia.

\subsection{Kiinteä liukuhihna}

Ennen ensimmäisiä \emph{grafiikkakiihdyttimiä}, joihin kuului muun muassa 3Dfx:n Voodoo, grafiikan piirto tapahtui prosessorilla, jonka työtaakkaa erilliset grafiikkakiihdyttimet kehitettiin vähentämään. Vapaasti ohjelmoitavia sävyttimiä ei vielä tuolloin kuitenkaan ollut, ja grafiikkaa tuotettiin käyttämällä hyväksi grafiikkapiirien \emph{kiinteää liukuhihnaa} (Fixed-Function Pipeline). Kehittäjä saattoi siis antaa raskaat laskutyöt grafiikkakiihdyttimelle hoidettavaksi käyttäen kiinteää liukuhihnaa, mutta itse liukuhihnan suorittamiin funktioihin ei voinut puuttua muuten kuin parametrien avulla. Kiinteä liukuhihna näytönohjaimessa nopeutti laskentaa ja toi mukanaan mahdollisuuksia luoda standardioperaatioiden rajoissa graafisia tehokeinoja ja efektejä, esimerkiksi Goraud-sävytyksen, josta on esimerkki kuvassa \ref{FGP}. Verrattuna monikulmion tasaiseen väritykseen Gouraud-sävytyksessä kärkipisteiden väriarvot interpoloidaan monikulmion sisällä, jolloin saadaan tasaisempi ja luonnollisemman näköinen heijastus.

\begin{figure}[!htpb]
\includegraphics[width=\textwidth]{/Users/jannetim/Documents/SeminaariF2015materials/gouraud.png}
\caption{Tasainen ja Gouraud-sävytys}
\label{FGP}
\end{figure}

Myöhemmin tulivat ensimmäiset \emph{ohjelmoitavaa grafiikkaliukuhihnaa} tukevat näytönohjainpiirit, joissa kiinteän liukuhihnan pystyi korvaamaan omilla vapaasti ohjelmoitavilla sävyttimillä. Korvaamalla kiinteän liukuhihnan laskenta nykyajan grafiikkapiirien tukemilla ohjelmoitavilla sävyttimillä voidaan vapaasti muokata sävyttimien käyttäytymistä ja saavutetaan mahdollisuus piirtää kuvaa enemminkin luovuuden rajoissa kuin ennaltamääriteltyjen ehtojen. 

\subsection{Varhaiset ohjelmoitavat sävyttimet}

Ensimmäiset ohjelmoitavan liukuhihnan sävytinmallit tukivat ainoastaan alemman tason konekieliin pohjautuvilla kielillä ohjelmointia. Tällaiset kielet lainasivat periaatteensa prosessoriarkkitehtuurista, jossa konekielen käskykanta on yksinkertainen ja rajoitettu. Sen lisäksi, että kehittäjien täytyi luoda sävytin, täytyi se luoda lisäksi usein erikseen sekä OpenGL- että Direct3D-rajapinnoille johtuen näiden kahden suosituimman rajapinnan kielien poikkeavuuksista . Aluksi laitteet tukivat ainoastaan kärkipiste- ja pikselisävytintä, laitekohtaisesti jopa vain ensimmäistä näistä. Silloiset rajoitukset konekäskyjen määrässä ja rekistereiden koossa tekivät silti vielä joistain asioista haasteellisia toteuttaa. Esimerkiksi pikselikohtainen valaistus, jonka nykyään voi ajatella olevan suoraviivaisesti toteutettavissa, oli tuolloin hankala toteuttaa laitteiston rajoituksista johtuen ja vaati kehittäjiltä erilaisia keinoja toimiakseen reaaliajassa \cite[s. 174-176]{She08}.

Eräs edelläkävijöitä sävyttimien saralla oli tietokoneanimaatioelokuvistaan tunnettu Pixar-yhtiö jo 1980-luvun lopussa kehittämällään \emph{RenderMan}-kielellä, jota on käytetty satojen elokuvien visuaalisiin efekteihin ja animaatioelokuvien piirtämiseen \cite{Pix15}.

\subsection{Korkean tason sävytinkielet}

Ohjelmoitavien sävyttimien alkuaikojen jälkeen alettiin sävyttimien luomisessa siirtyä alemman tason kielistä erillisiin korkean tason sävytinkieliin. Alemman tason kieliin verrattuna korkean tason kielillä on useita hyötyjä, ja ne pätevät myös korkean tason sävytinohjelmoinnissa: helpompi luettavuus, kirjoitettavuus, muokattavuus, virheiden etsintä ja löytäminen sekä yleisesti kehitysvauhdin nopeus \cite[s.183-185]{She08}. Ohjelmoitavien sävyttimien tultua kasvoi myös tarve korkean tason sävytinkielille, joista mainittavimpina muodostuivat C-pohjaiset Nvidian Cg- (C for Graphics), \cite{Nvi03} Microsoftin HLSL- (High Level Shading Language) ja OpenGL:n \cite{Khr15} GLSL. Cg ja HLSL syntyivät Microsoftin ja Nvidian yhteistyöprojektin tuloksena, ja ovatkin lähes identtiset \cite[s. 198]{She08}. Näistä kielistä Cg on jo vanhentunut, eikä sitä enää päivitetä \cite{Nvi12}. HLSL ja GLSL ovat monin osin samankaltaisia, ja ne tukevat samanlaisia sisäänrakennettuja funktioita. Eroina on muun muassa joidenkin samojen datatyyppien poikkeavat nimeämiskäytännöt. Esimerkiksi neljän komponentin vektori on GLSL-kielellä ilmaistuna \texttt{vec4}, kun taas HLSL-kielellä se on \texttt{float4} \cite[s. 198]{She08}. Myöhemmät koodiesimerkit ja käytetty sävytinterminologia ovat pääasiassa Direct3D:n ja HLSL:n käytäntöjen mukaista.

\section{Ohjelmoitavat sävyttimet}

Ohjelmoitavien sävyttimien osalta grafiikkaliukuhihna rakentuu pääpiirteissään \emph{kärkipiste}-, \emph{tesselaatio}, \emph{geometria}- ja \emph{pikseli}sävyttimestä. Yksi sävytinvaihe ottaa syötteenään vastaan edellisen tulosteen, joten jokainen vaihe voi jatkaa seuraavan datan työstämistä, kun on saanut edellisen työn valmiiksi. Tämä grafiikkaliukuhihnan malli mahdollistaa sävytinprosessoinnin rinnakkaistamisen erittäin hyvin \cite{Ake02}. 

Tesselaatiosävytin ja geometriasävytin ovat uudempia tulokkaita, ja niiden käyttö ei ole pakollista, jolloin niiden toiminnallisuus voidaan jättää pois tai korvata muilla tavoin. Esimerkiksi ennen tesselaatiolle omistettua omaa sävytintä tesselaatiosävytys toteutettiin geometriasävyttimen avuin \cite{Sch14}. Tässä luvussa tarkastellaan erilaisia ohjelmoitavia sävyttimiä siinä järjestyksessä, jossa ne toimivat grafiikkaliukuhihnalla.

\subsection{Kärkipistesävytin}

\emph{Kärkipistesävytin}, tai \emph{verteksisävytin}, ajetaan kerran jokaista monikulmion, tarkemmin kolmion, kärkipistettä kohden. Kärkipistesävytin ottaa syötteenään kärkipisteen \emph{attribuuttitiedon}, joka sisältää muun muassa kyseisen kärkipisteen sijainnin malli- tai maailmakoordinaatistossa, sekä pinnan normaalivektorin. Tulosteena kärkipistesävytin antaa yhden kärkipisteen, joka on käynyt läpi valaistuksen ja koordinaatiston muunnosvaiheet, ja joka ilmaistaan \emph{normalisoidussa kuvausavaruudessa}. Yleisesti siis kärkipistesävytin voi muokata monikulmion kärkipisteen, normaalin, tekstuurikoordinaattien ja paikan arvoja. Sävyttimellä voi luoda esimerkiksi tuulessa heiluvat puiden oksat tai vedenpinnan väreilyn.

Kärkipistesävyttimen tärkeimpiin tehtäviin kuuluu tehdä tarvittavat koordinaatistomuunnokset syötteinä saaduille kärkipisteille. Ensimmäisenä tehdään \emph{mallimuunnos}, eli muunnos mallikoordinaatistosta maailmakoordinaatistoon, jolloin kärkipiste sidotaan omasta paikallisesta koordinaatistostaan maailmanäkymään. Tämän jälkeen tehdään \emph{katsojamuunnos} maailmakoordinaatistosta katsojan koordinaatistoon, jolloin origo on katsotaan ``kameran'' näkökulmasta. Viimeiseksi suoritetaan \emph{projektio-} tai \emph{perspektiivimuunnos}, jolloin tuotetaan renessanssiajan kuvataiteesta tuttu luonnollisen elämän persepektiiviä jäljittelevä kuva, jossa kauempana olevat kohteet näkyvät pienempinä \emph{näköfrustumissa}. Näköfrustumi on suorakulmainen kuvausikkuna, joka projektion keskipisteestä (katsojan silmästä) lähtien muodostaa äärettömyyteen jatkuvan pyramidin, jonka sisällä katsojalle näkyvät asiat ovat. Tätä tilaa kutsutaan normalisoiduksi kuva-avaruudeksi \cite{Puh08}. Vähintään kärkipistesävyttimen tulee siis antaa tuloksena kärkipiste muunnettuna normalisoituun kuva-avaruuteen \emph{uniformina} eli vakiomuotoisena, kaikille sävyttimille yhteisenä tietona \cite{Puh08}.

Kärkipistesävytin on pakollinen vaihe grafiikkaliukuhihnalla, ja sen täytyy vähintään kuljettaa lävitseen syötteenä saatu kärkipiste, vaikkei laskusuorituksia tehtäisikään. Alla HLSL-kielellä kirjoitettu yksinkertainen kärkipistesävytin syötteineen ja tuloksineen \cite{Mic11}:
\lstset{breaklines=true}
\begin{lstlisting}[language=c, basicstyle=\footnotesize]
// Input / Output structures

struct VS_INPUT
{
    float4 vPosition    : POSITION;
    float3 vNormal        : NORMAL;
    float2 vTexcoord    : TEXCOORD0;
};

struct VS_OUTPUT
{
    float3 vNormal        : NORMAL;
    float2 vTexcoord    : TEXCOORD0;
    float4 vPosition    : SV_POSITION;
};

// Vertex Shader

VS_OUTPUT VSMain( VS_INPUT Input )
{
    VS_OUTPUT Output;
    
    Output.vPosition = mul( Input.vPosition, g_mWorldViewProjection );
    Output.vNormal = mul( Input.vNormal, (float3x3)g_mWorld );
    Output.vTexcoord = Input.vTexcoord;
    
    return Output;
}
\end{lstlisting}

Syöte- ja tulos-structeissa on määritelty vektoreina (\texttt{float}$n$) kärkipisteen sijainti, normaali ja tekstuurikoordinaatti. \texttt{VSMain}-metodi ottaa syötteenään kyseiset arvot  ja tekee sijainnille sekä normaalille matriisimuunnokset. Sijainnille tehdään maailma-katsojamuunnos katsojan koordinaatistoon ja normaalille maailmamuunnos. Tekstuurikoordinaatti kulkee esimerkissä sävyttimen läpi muuttumatta. Vaikka esimerkki on yksinkertainen, noudattelee sävyttimien ohjelmointi samaa logiikkaa grafiikkaliukuhihnan kaikissa vaiheissa.

\subsection{Tesselaatiosävytin}

Tutkielman sävyttimistä uusin on Shader Model 5.0:n, rajapintojen kohdalla DirectX 11 ja OpenGL 4.0, myötä tullut tesselaatiosävytin. \emph{Tesselaatiossa} \emph{monikulmioverkko} (polygon mesh) eli kärkipisteistä ja reunaviivoista kohtaava kolmioiden joukko\cite{Puh08}, jaetaan pienempiin osasiin, kuten vaikkapa kolmio kahteen pienempään kolmioon \cite{Nvi10}. Yksinkertaisesti siis tesselaatio on monikulmioiden rikkomista ja jakamista pienempiin ja hienompiin osasiin. 

Tesselaatiosävytin on jaettu kolmeen vaiheeseen, joista ensimmäinen ja kolmas ovat ohjelmoitavia sävytinvaiheita. Vaiheet alkaen ensimmäisestä ovat \emph{Hull Shader}, varsinaisen työn tekevä kiinteä tesselaatiovaihe\emph{Tessellation Stage} ja \emph{Domain Shader} \cite{Mic11}. OpenGL-terminologiassa vastaavat vaiheiden nimet ovat Control, Primitive Generator ja Evaluation. Tesselaatiossa ensimmäisen sävytinvaiheen tehtävä on määrittää paljonko syötettä tesseloidaan. Kiinteä vaihe hoitaa tämän jälkeen laskutyön saamansa syötteen perusteella, minkä jälkeen kolmannen vaiheen sävytin työstää tesseloidun datan ja määrittää kärkipisteiden sijainnit. Käytännössä tesselaation kolmas vaihe toimii kuten tavallinen kärkipistesävytin.

Menetelmänä tessellointi ei välttämättä tuo suoraan ulkonäöllisiä mullistuksia esimerkiksi pelien ulkonäköön, sillä ulkoasun kannalta ei ole merkitystä onko esimerkiksi neliö piirretty kahden vai kymmenien kolmioiden avulla. Sen sijaan yhdistämällä tesselaatioon muita tekniikoita, ja laittamalla monikulmioista pilkotut palaset esittämään uutta informaatiota, saadaan piirtoa monimutkaisemmaksi ja realistisemmaksi \cite{Nvi10}. Esimerkiksi monimutkaisen geometrian luominen lennosta on ennen tesselaatiosävytintä ollut vaikeata, kun työ on tehty ensin prosessorilla ja ladattu sen jälkeen grafiikkapiirille. Shader Model 5 -version myötä grafiikkapiirien tukiessa tesselaatiota laitetasolla voidaan tesselointi tehdä tehokkaasti \cite{Sch14}.

Eräs tesseloinnin tekniikka on \emph{nyrjäyttäminen} (Displacement mapping), jossa tesseloidun pinnan kärkipisteitä nostetaan tai lasketaan korkeusattribuutin perusteella, jolloin saadaan luotua epätasaisia pintoja \cite{Nvi10}. Nyrjäytystekniikkaa havainnollistaa kuva \ref{displacement}. Tessellointi säästää myös muistia ja kaistanleveyttä mahdollistamalla yksityiskohtaiset pinnat pieniresoluutioisilla tekstuureilla \cite{Mic11} \cite{Nvi10}.

\begin{figure}[t]
\includegraphics[width=\textwidth]{/Users/jannetim/Documents/SeminaariF2015materials/displacementmapping.jpg}
\caption{Nyrjäytyskartan realistisemmat pintaerot}
\label{displacement}
\end{figure}

Tesselaation avulla saavutettava keino on myös 3D-mallien silottaminen normaalipiste-kolmioiden (PN-triangle) avulla \cite{Vla01}. Esimerkki kuvassa \ref{SCoP} näkyvän Stalker: Call of Pripyat -pelin hahmon kaasunaamarin suodattimen reunoja on saatu tesselaatiolla luonnollisemmalla tavalla kaartuviksi (DX11) verrattuna yksinkertaisempaan ja vähäisemmillä monikulmioilla piirrettyyn malliin (DX10). 

\begin{figure}[!hbpt]
\includegraphics[width=\textwidth]{/Users/jannetim/Documents/SeminaariF2015materials/stalkercop-dx11.jpg}
\caption{Monikulmiomallin silottaminen tesselaation avulla pelissä Stalker: Call of Pripyat}
\label{SCoP}
\end{figure}

\emph{Dynaamisella tesselaatiolla} voidaan esimerkiksi skaalata mallien piirron tarkkuutta näkyvyyden suhteen muuttamalla yksityiskohtien määrää lennosta \cite{Nvi10}. Tällöin esimerkiksi avarassa ulkoilmapelinäkymässä kaukaa katsottuna jostain mallista piirretään malli muutamalla monikulmiolla, ja lähestyttäessä monikulmioiden määrää lisätään dynaamisesti, kunnes läheltä katsottuna malli voidaan piirtää tuhansista monikulmioista \cite{Gre14}. Dynaamisella tesselaatiolla voidaan estää objektien yhtäkkinen tyhjästä näkyviin -tyylinen piirtäminen, jossa yksityiskohtainen malli täytyy piirtää kerralla kokonaisuudessaan.

\subsection{Geometriasävytin}

Geometriasävyttimet ovat kärkipiste- ja pikselisävyttimiin verrattuina uudempi tekniikka sen tultua esitellyksi Shader Model 4.0:n ja DirectX 10:n myötä. Geometriasävytin sijaitsee grafiikkaliukuhihnalla tesselaatiosävyttimen jälkeen ja ennen pikselisävytintä. Geometriasävytin on vaihtoehtoinen osa liukuhihnaa.

Geometriasävytin ottaa syötteenään \emph{n}-kärkipisteestä muodostuvia primitiivejä, kuten pisteitä (n=1), suoria (n=2) tai kolmioita (n=3).  Syötteistä geometriasävytin muokkaa, valikoi ja jopa luo uusia primitiivejä \cite{Gre14}. Sävytin voi siis antaa tuloksena syötteestä riippumattoman määrän geometrisiä primitiivejä, ja jotka eivät välttämättä ole samaa tyyppiä kuin syötteenä saadut. Geometriasävytin voi esimerkiksi yhdistellä yksittäisiä kärkipisteitä, koota kolmioita uudeksi monikulmioksi, tai hylätä syötteenä saadun primitiivin kokonaan. Toisin kuin kärkipistesävytin, joka kykenee käsittelemään vain yhden monikulmion kärkipisteen kerrallaan, on geometriasävyttimellä täydelliset tiedot käsittelemästään primitiivistä \cite{Khr15}. Täten se pystyy käsittelemään useista kärkipisteistä koostuvaa primitiiviä \cite{Mic11}.

Geometriasävyttimellä tyypillisesti luotuihin asioihin kuuluvat esimerkiksi \emph{partikkelijärjestelmät} ja \emph{mainostaulut} \ref{billboard}. Partikkelijärjestelmillä voidaan luoda partikkeliefektejä kuten tuli, savu ja kipinät. Efektien luomisessa käytetään pieniä geometrisiä objekteja, kuten kahdesta kolmiosta koostuvia nelikulmioita. Partikkelijärjestelmä voi esimerkiksi määritellä yksittäisen partikkelin eliniän tietyn aikaikkunan sisällä esimerkiksi kipinäryöpyn tapauksessa. Partikkeliobjektien ollessa litteitä, ovat ne usein mainostaulutettu eli partikkelipinnan normaali osoittaa aina koordinaatiston kameran suuntaan. Esimerkiksi katsojalle näkyvä linssiheijastus katsottaessa valonlähdettä ensimmäisen persoonan peleissä on tästä hyvä esimerkki. Mainostaulutuksesta toinen hyvä esimerkki ovat kaksiulotteiset \emph{spritet} eli 3D-maailmaan sijoitettavat bittikartat \cite[s. 216-217]{Puh08}.

\begin{figure}[!htpb]
\includegraphics[width=\textwidth]{/Users/jannetim/Documents/SeminaariF2015materials/events.jpg}
\caption{Partikkelijärjestelmä ja mainostauluttaminen}
\label{billboard}
\end{figure}

Ennen tesselaatiosävyttimen tuloa geometriaävyttimellä hoidettiin myös esimerkiksi aiemmin esitelty dynaaminen tesselaatio \cite{Mic11}. Vaikka geometriasävyttimellä onkin mahdollista tehdä yksinkertaista tesselaatiota, se muodostuu grafiikkaliukuhihnalle helposti pullonkaulaksi \cite{Sch14}. Geometriasävyttimen parhaat käyttötapaukset ovat esimerkiksi sellaisia, joissa sävytin voi luoda yhdestä syöteprimitiivistä useita uusia primitiivejä.

\subsection{Pikselisävytin}

Pikselisävytin (OpenGL-terminologiassa fragmenttisävytin) on grafiikkaliukuhihnalla kiinteän rasterointivaiheen jälkeen, ja se laskee muunnoksia pikselikohtaisesti. Rasteroitujen primitiivien jokaiselle pikselille kutsutaan erikseen pikselisävytintä, joka laskee pikselille väriarvon ja Z-syvyyden. Pikselin väriarvo sekä Z-syvyys lasketaan syötteenä saatun vektorimuotoisen datan, kuten normaalivektorin, värin, tekstuurikoordinaattien, interpoloitujen valonlähteiden suuntien ja katsojan suunnan, perusteella. Erityisesti pikseliin kohdistuva valaistuksen laskenta voidaan laskea näiden tietojen perusteella \cite{Puh08}.

Monet näyttävät tietokonepeleissä käytettävät tehostekeinot, kuten pinnan kuhmutus tai metallipinnoista heijastuvan valon käyttäytymistä mallintava \emph{Fresnel-heijastus} \cite{Laz05}, luodaan juuri pikselisävyttimien tasolla. Myös Gouraud-sävytystä realistisempi Phong sävytys luodaan pikselisävyttimellä. Phong-sävytyksen eri komponentit ovat esitelty kuvassa \ref{Phong}. Pikselisävyttimen tärkeimpiin tehtäviin lukeutuvatkin teksturointi ja valaistuksen laskenta.

\begin{figure}[!htbp]
\includegraphics[width=\textwidth]{/Users/jannetim/Documents/SeminaariF2015materials/Phong_components.png}
\caption{Phong-sävytys}
\label{Phong}
\end{figure}

Eräs tietokonepeleissäkin hyödynnetty keino on pikselisävyttimellä luotava \emph{Cel-sävytys}. Cel-sävytyksessä  ... .... .... ..... . Valaistuksessa pikseleiden väriarvot lasketaan katsojan suunnan ja pinnan normaalivektorin avulla siten, että väriarvo muuttuu portaittain sitä tummemmaksi mitä enemmän pinta osoittaa pois katsojasta. Cel-sävytyksestä on esimerkki kuvassa \ref{Cel}.

\begin{figure}[!htbp]
\includegraphics[width=\textwidth]{/Users/jannetim/Documents/SeminaariF2015materials/Celshading_teapot.png}
\caption{Cel-sävytys}
\label{Cel}
\end{figure}

\subsection{Laskentasävytin}

Shader Model 5.0 toi myös tuen \emph{laskentasävyttimelle} (englanniksi Compute Shader). Laskentasävyttimen on enemmän yleiskäyttöinen ja löyhästi määritelty sävytin kuin muut tutkielmassa esitellyt sävyttimet. Laskentasävytin ei varsinaisesti kuulu grafiikkaliukuhihnalle, eikä sen syötteitä ja tuloksia ole samalla tavoin tarkasti määritelty kuin muilla sävyttimillä \cite{Kes14}. Laskentasävytin voi siis ottaa syötteenään muutakin dataa kuin grafiikkalaskennassa käytettyjä objekteja kuten kolmioita. Eräs käyttötarkoitus laskentasävyttimelle onkin \emph{GPGPU}, eli yleiskäyttöinen laskenta, jolloin näytönohjainten rinnakkaisuutta hyödynnetään erilaisten raskaiden laskutöiden suorituksessa \cite{Uni15}. Tietokonepelien kohdalla laskentasävytintä voidaan hyödyntää esimerkiksi fysiikan mallinnuksen tai tekoälyn toiminnan laskemisessa \cite[Luku: Compute Shader Overview]{Mic11}. Laskentasävyttimille on olemassa myös omia ohjelmointirajapintoja, joihin kuuluvat muun muassa Nvidian CUDA, Khronos Groupin OpenCL sekä Microsoftin DirectCompute.

\section{Sävytinohjelman luominen ja käyttäminen}

Grafiikkapiirit tukevat laitteistotasolla eri sävytinvaiheita, joten sävytin täytyy ensin ohjelmoida ja sen jälkeen ladata grafiikkapiirille suoritettavaksi grafiikkaliukuhihnalla.

Sävyttimen luominen. Esimerkiksi käyttäen hyväksi HLSL-kieltä tapahtuisi sävyttäjän luomisprosessi näin...HLSL:n tapauksessa sävyttimen voi kääntää ennakkoon tai lennosta ajonaikana. Tässä luvussa käydään läpi esikäännetyn sävyttimen käyttöönoton päävaiheet \cite{Mic15}.

Kirjoitettu .hlsl -sävytintiedosto tulee ensin kääntää sävytinobjektiksi, eli HLSL:n tapauksessa .cso-tiedostoksi (complied shader object). Kääntämiskutsun voi HLSL:n tapauksessa tehdä suoraan Microsoftin Visual Studio -kehitysympäristössä käyttäen fxc.exe HLSL-kääntäjää. Itse kääntökutsu on pääpiirteissään seuraavanlainen:

\lstset{breaklines=true}
\begin{lstlisting}[language=c++, basicstyle = \small]
D3DCompileFromFile( srcFile, defines, D3D_COMPILE_STANDARD_FILE_INCLUDE, entryPoint, profile, flags, 0, &shaderBlob, &errorBlob );
\end{lstlisting}

Oleellista kutsussa ovat parametrit \emph{srcFile} eli lähdetiedosto, sekä \emph{defines}, jonka tulee viitata sävytinfunktion nimeen. Lähdetiedostona voisi olla esimerkiksi kärkipistesävytin VSexample.hlsl ja sävytinfunktiona tiedoston koodista löytyvä VSmain. Muista kääntöfunktion parametreista ei tarvitse tässä esimerkissä välittää.

Kun sävytin on ladattu ja käännetty, kapsuloidaan se sävytinobjektiksi.

\lstset{breaklines=true}
\begin{lstlisting}[language=c++, basicstyle = \footnotesize]
ID3D11VertexShader  *pVertexShader
ID3DBlob pBlob;

pd3dDevice->CreateVertexShader( pBlob->GetBufferPointer(), pBlob->GetBufferSize(), &ppVertexShader );

\end{lstlisting}

Funktiokutsun ensimmäinen parametri on blob-muotoinen muuttuja, jossa käännetty data sijaitsee. Toinen parametri on datan koko ja kolmas on sävytinobjektin sijainti.

Lopuksi käytetään setShader-funktiota sitomaan sävytinohjelma grafiikkaliukuhihnalle.

\lstset{breaklines=true}
\begin{lstlisting}[language=c++, basicstyle = \footnotesize]
ID3D11VertexShader *pVertexShader
g_d3dDeviceContext->VSSetShader( PVertexShader, nullptr, 0 );
\end{lstlisting}


\section{Yhteenveto}

Tutkielmassa esiteltiin 3D-grafiikassa ja erityisesti tietokonepelien grafiikan piirtämisessä käytettäviä ohjelmoitavia sävyttimiä, niiden historiaa ja mihin niitä voi käyttää. Sävyttimet muodostavat omat moduulinsa grafiikkaliukuhihnalla, joka on vaiheittainen prosessi ruudulla näkyvän grafiikan tuottamiseksi. Ohjelmoitavat sävyttimet ovat ohjelmia, jotka suoritetaan grafiikkalaitteisto tukeman grafiikkaliukuhihnan tietyissä vaiheissa.

Kärkipiste- ja pikselisävytin ovat perinteisiä ja sävyttimiä, joita on käytetty 3D-grafiikan tehokeinojen perustana jo pitkään. Uudempia tulokkaita ovat geometriasävytin ja viimeisinpänä tesselaatiosävytin, joista etenkin viimeinen vaikuttaisi tarjoavan monia uudenlaisia keinoja luoda entistä realistisemman näköistä grafiikka reaaliajassa.

Grafiikkaliukuhihnan järjestyksessä kärkipistesävyttimen tärkein tehtävä on laskea monikulmioiden kärkipisteille koordinaatistomuunnokset. Shader Model 5.0:n myötä laitteistotasolla tuettu tesselaatiosävytin on tehokas työkalu lisäyksityiskohtien luonnissa. Tesseloinnissa monikulmiot voidaan jakaa useisiin pienempiin osasiin, joita voidaan käyttää hyväksi uudenlaisten visuaalisten efektien, kuten nyrjäytyskartan, tekemisessä. Geometriasävytin puolestaan voi antaa tuloksenaan syötteestään riippumattoman määrän primitiivejä. Saamastaan syötteestä se voi luoda, valikoida, muokata tai hylätä tarpeelliseksi katsotun määrän tulosprimitiivejä. Tesselaatio- ja geometriasävytin ovat vaihtoehtoisia vaiheita grafiikkaliukuhihnalla, ja ollessaan käytössä ne tekevät osin samoja tehtäviä kuin kärkipistesävytin, esimerkiksi kärkipisteiden sijaintiin liittyvissä laskutoimituksissa. Näiden vaiheiden jälkeen data siirtyy rasterointivaiheeseen, jossa rajatun kokoiseen kaksiulotteiseen piirtoikkunaan muodostetaan kuva pikseleistä. Rasteroinnin jälkeen työvuorossa on pikselisävytin.



% --- References ---
%
% bibtex is used to generate the bibliography. The babplain style
% will generate numeric references (e.g. [1]) appropriate for theoretical
% computer science. If you need alphanumeric references (e.g [Tur90]), use
%
% \bibliographystyle{babalpha-lf}
%
% instead.

%\bibliographystyle{babalpha-lf}
\bibliographystyle{babalpha-lf}
\bibliography{references-fi}


% --- Appendices ---

% uncomment the following

% \newpage
% \appendix
% 
% \section{Esimerkkiliite}

\end{document}